\documentclass{article}
\usepackage[utf8]{inputenc}

\title{RIOS Patent Classification and Processor Patent Analysis for Robust RISC-V Ecosystem}
% \author{Lei Ren}
% \date{September 2022}

\begin{document}

\maketitle

\section{Proposal}

RISC-V has driven needs for its industrialization, whereas potential patent infringement and implement litigation may occur at processor microarchitecture. Thus, a secured legal structure against patent troll and pitfalls contributes largely to the growth of RISC-V, and it is one of the key tools our RIOS Lab is building. Here we analyzed nearly 5000 patents from MIPS and ARM, and established a new classification system, RPC, in favor of CPU domain knowledge which existing systems IPC and CPC are lacking, based on different technical and functional modules during processor design and development, statistical landscape results of which outshined those based on IPC and CPC. For healthy RISC-V community, legal status, patent assignment, active patent owners, application and grant rate, imminent expiration, number of citations and claims, and litigations were all investigated. Our research indicated CPU tech areas of microarchitecture and ISA have drawn most ambiguities. Through this work RIOS Lab would dedicate a defensive and monitoring system against potential IP risks for RISC-V processor implementations, and make patent freedom and open source safely achieved in RISC-V ecosystem.

\end{document}
