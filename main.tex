\documentclass{article}
\usepackage[utf8]{inputenc}

\title{RIOS Patent Classification and Processor Patent Analysis for Robust RISC-V Ecosystem}
% \author{Lei Ren}
% \date{September 2022}

\begin{document}

\maketitle

\section{Proposal}

Tremendous attractions for RISC-V have driven needs for its effective adoption by industry. However, during its development, potential patent infringement and implement litigation might occur at processor architecture and microarchitecture. Thus, a secured legal infrastructure against patent troll and pitfalls contributes significantly to the growth of RISC-V community, and it’s also one of the key tools our RIOS Lab is building. Here we systematically analyzed nearly 5000 patents from MIPS and ARM, and established a new classification system, RPC (RIOS Patent Classification), in favor of CPU domain knowledge which the existing systems IPC and CPC are lacking, based on different technical and functional modules during processor design and development process, statistical landscape analysis results of which outshined those based on IPC and CPC. For a healthy RISC-V community, legal status, patent assignment, active patent owners, application and grant rate, imminent expiration, number of citations and independent claims, and litigation cases were all investigated. Our research indicated CPU tech areas of microarchitecture and ISA have drawn most of the ambiguities. Through this work RIOS Lab would dedicate a defensive and monitoring system against potential IP risks for RISC-V processor implementations, and make patent freedom and open source safely achieved in RISC-V ecosystem.

\end{document}
